


\chapter{Literature Review}
This chapter is a review of the relevant literature around deep learning and how this type of machine learning has advanced in the past years . In particular, the report will discuss how deep learning can be used in the diagnosis of pathologies in chest X-rays. As well as this we will review successful projects from the past that are closely linked to this thesis.

\section{Overview}

Radiology is a branch of medicine which uses medical imaging technology to generate images so that radiologists can view structures within the human body. The types of images range from X-rays, CT scan(computed tomography scan) and ultrasounds. \cite{bradley2008history}. For the past few years, radiology in the UK  has seen disruption in the National Health Service. A reoccurring issue amongst radiology departments is lack of staff to interpret scans. According to the Royal College of Radiologists, the NHS does not have enough radiologists to meet diagnostic imaging demands leaving patients at risk. \cite{rimmer2017radiologist}. 
\space
\subsection{Faster Diagnosis}
One area which can augment the diagnosis of medical scans is image classification. This process of image classification involves teaching a model to learn features from images and then test on unseen images to compute accuracy. In recent times, image classification through deep learning methods has produced  superhuman performance on several image-based classification tasks\cite{krizhevsky2012imagenet}.