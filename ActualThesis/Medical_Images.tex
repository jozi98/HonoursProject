\section{Medical Images}

As mentioned in the Overview section, radiology departments in the UK have struggled to meet imaging diagnostic demands. The survey published by the Royal College of Radiology reported 97\% of radiology departments were unable to do so within their working hours. As mentioned in the paper, ``It points to an insufficient number of radiologists to meet the increasing demand for imaging and diagnostic services". According to the report it cited that radiology has the second lowest proportion of trainees to consultants: 26 trainees to every 74 consultants. This is compared to an average in all specialities of 40 trainees for every 60 consultants. The report also cites that there was a particular workforce shortage prominent in Scotland, where the consultant workforce grew by 7\% from 2010 to 2016 but demand for CT,MRI scans increased by 10\%. 
One of the profound impacts of such shortage is the risk to patient care as well as economical effects. The NHS paid nearly £88m in 2016 for backlogs of radiology examinations to cover backlogs of radiology examinations to be reported. To cover these backlogs 92\% of radiology departments paid staff overtime, 78\% outsourced to private companies and 52\% employed ad hoc locums. This amount could have paid for at least 1028 full time radiology consultants.\cite{rimmer2017radiologist}

Given this what has been done to augment the analysis of medical images in the NHS and reduce extreme workload and large expenses ? One particular software which has bee used in the clinical setting is Computer Aided Diagnosis(CAD). CAD is technology which is designed to decrease observational oversights and false negative rates for physicians interpreting medical images such as mammographies \cite{castellino2005computer}. According Dr. Paul Chang, the reason this type of software did not receive as much attention, compared to the possibility of using deep learning, was that it used traditional methods of machine learning. \cite{youtube}

\subsection{Challenges}

There many challenges associated when applying new technologies in a clinical setting. Medical images are very large and complicated and for most of the image classification tasks in the ImageNet example , distinct features or pixels can be learned by a model. Dr. Matthew Lungren , radiologist at Stanford Medical Center, mentioned how most of the pixels in a chest X-Ray from a healthy person who happens to have lunge cancer , will tell you that the image is of a healthy person. But there is only a small proportion of the image which indicates the finding of lung cancer fo the classifier. Dr Lungren also noticed how medical images may have, what is known as artefacts or tokens in the image. Doctors are able to deal with context when looking at an image , but such artefacts or tokens may lead to a model making a decision with higher degree of confidence but in actual fact the classification is incorrect \cite{DrLungren.2018}

