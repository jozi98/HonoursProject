\section{Summary of Review}

From the review, it is quite evident that with the ever-growing concern in the NHS of a shortage of radiologists \cite{rimmer2017radiologist}, solutions need to be found in order to solve the issue of increased backlogs. Image classification, as mentioned, can provide augmentation of diagnosis of X-rays. Great leaps and bounds have been made in the field of deep learning and particularly well know CNN architectures show performance exceeding that of human intelligence in the case of ImageNet challenge \cite{krizhevsky2012imagenet}s. As well as this more sophisticated example of applying deep CNN's to medical images has shown outstanding results on pathologies such as Pneumonia and Cardiomegaly \cite{rajpurkar2017chexnet}. This problem of diagnosing X-rays also comes with its challenges which have been outlined in the challenges section. Other hurdles must be kept in mind for implementation purposes, and these include making sure that pixel values are not lost. Given the nature of X-rays, pixel loss can lead to pathologies being removed from an image such as lung nodules and according to the Royal  College of Radiologists certain rules needs to be adhered to so that images do not lose quality when it comes to diagnosing them. Lastly, when training a model on medical images, the dataset chosen needs to be balanced in class labels. As well as this one of the big obstacles is lack of datasets of reliable labelling of datasets as noise in class labels can lead to reduced performance. 


\section{Conclusion}

In this section, we will conclude how the rest of this thesis will be implemented and how the proposed problem in the introduction will be approached. From the findings above the aim is to apply one of the well known CNN architectures to build an accurate image classification pipeline on predicting the presence of Pneumonia in chest X-rays. In order to train the model, a carefully curated dataset must be chosen.  The dataset that will be used in this thesis comes from a public dataset carefully curated by Stanford ML Group called CheXPert. This dataset was created to solve three problems which include enough instances, strong reference standards and provide human performance metrics for comparison. It consists of 224,316 chest X-rays, and one of the main obstacles in the development of these datasets is the lack of strong radiologist-annotated ground truth.
The objectives is to build an accurate model which can predict and localise the Pneumonia with techniques such as occlusion. Another objective would be to build an interface which allows the entry of a chest X-ray and a predicted class label will be given back as well as a heat-map indicating which parts of the image had the highest activation. 
Lastly, if results from the initial pipeline show high accuracy, specificity, sensitivity and AUC, then the model could be extended to classify more than one pathology contained in the dataset.

