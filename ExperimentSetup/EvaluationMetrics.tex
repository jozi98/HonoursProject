
\section{Going Beyond Accuracy}\label{metrics}


As shown in figure \ref{fig:TestingDist}, the class distribution of the testing set is imbalanced. Due to this imbalance, it is important to look beyond accuracy to validate a model performance. For example, since the test set is imbalanced, if all the test cases were labelled positive, i.e. a class label of pneumonia, then the accuracy on the test set would be 63\%. However, the accuracy does not show the underlying performance of the model, which indicates that it is classifying every patient as having the infection.


Due to the domain of this paper and its aim: Automatic classification of X-ray Images as either normal or abnormal, a confusion matrix needs to be considered when evaluating results. Figure \ref{fig:ConfMatrix} shows an example of a confusion matrix that will be used to analyse results. The number of pathological samples that are \textit{correctly} identified as a pathological sample by the model is called a true positive (TP). The number of pathological samples that are \textit{incorrectly} classified as normal by the model is called a false negative(FN). The number of normal samples that are \textit{correctly} identified as normal is called a true negative(TN) and in a similar fashion, the number of normal samples that are \textit{incorrectly} identified as pathological samples is called false positive(FP). 
Given the domain of this study, the cost of a false negative is higher than a false positive

	
 \begin{figure}[H]
	\centering
	\hspace{-2cm}
	\includegraphics[scale=0.6]{images/ConfusionMatrix}
	\caption{Confusion Matrix}
	\label{fig:ConfMatrix}
\end{figure}



Although accuracy is still considered, this study will also examine the values of precision, recall/sensitivity and specificity. Accuracy, specificity, recall and precision are computed as follows:

\begin{center}

\[Accuracy=\frac{TN+TP}{TN+FP+FN+TP}\]
\vspace{.5em}
\[Specificity=\frac{TN}{TN+FP}\]
\vspace{.5em}
\[Recall/Sensitivity=\frac{TP}{TP+FN}\]
\vspace{.5em}
\[Precision=\frac{TP}{TP+FP}\]
\end{center}







\begin{itemize}
	
	\item \textbf{Accuracy} is a basic measure shown as a percentage of the total correct classification out of all the samples in the testing set.
	\item \textbf{Specificity} shows the degree to which the model correctly identifies normal samples as normal. 
	\item \textbf{Recall/Sensitivity} is a measure which shows the degree to which the model does not miss a pathological sample
	\item \textbf{Precision} is measure of the exactness or quality of the model when it returns results for a specific class
	
	
\end{itemize}




