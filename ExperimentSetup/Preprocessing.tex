\subsection{Dataset Preprocessing}
% Oh hello there, here are some tasks: 
% 1. Remove 'due to the fact' on line 5 - its a filler-word
% 2. Change due to guidance on line 5

During exploration of the dataset it was shown that the X-ray images provided were of different sizes. This most likely due to different X-ray imaging machines used during collection of the dataset. Before applying the learning algorithm on the X-ray images we needed to make sure that each input image was of the same dimension. This is due to the way a CNN is configured via libraries like Keras, where the size of the input layer is constant. At first, all images were resized from their original size to 250 x 250. However, due to guidance, later experiments used a size of 256 x 256. 



\subsection{Normalising Images}

Data normalisation is an important step which ensures each pixel value for an image has a similar data distribution. This makes convergence faster while training a neural network. Although non-normalised data can be presented to the neural network, this can result in challenges during training, such as slower training of the model. 

As discussed in section \ref{sect:SotA}  ``State of the Art" , neural networks process inputs using small weight values which are multiplied by the input value. If these inputs are on a large unsigned integer scale, like pixel values, this may disrupt or slow down convergence of the training process. A pixel value of 0(white) compared to 255(black) are both equally as important but the 255 value will have a greater effect on the final prediction. This large value could cause the output to be wrong and lead to the training process taking longer. 

For the purposes of this paper, data normalisation is applied by dividing each pixel value by 255 and as a result all inputs to the model have a range between 0 and 1.

To better understand the stages of normalisation, figure \ref{fig:preprocessing} shows the steps taken to normalise each image.
\footnote[1]{\href{https://github.com/jozi98/HonoursProject/blob/master/ChestX-ray.ipynb}{Python Notebook:Preprocessing }} 

 \begin{figure}[H]
	\centering
	\includegraphics[scale=0.2]{images/Preprocessing.png}
	\caption{Stages of the Preprocessing and Normalisation}
	\label{fig:preprocessing}
\end{figure}


 
